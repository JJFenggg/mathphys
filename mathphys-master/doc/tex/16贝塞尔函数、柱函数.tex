\documentclass[11pt]{beamer}
\usepackage[UTF8,scheme=plain]{ctex}
\usepackage{listings}
\usepackage[utf8]{inputenc}
\usepackage[T1]{fontenc}
\usepackage{amsmath}
\usepackage{amsfonts}
\usepackage{amssymb}
\usepackage{mathrsfs}
\usepackage{graphicx}
\usetheme{Boadilla}

\usepackage{framed} % 可以用 \begin{shaded},即背景色块
\definecolor{shadecolor}{rgb}{0.9,0.9,0.9}

\newcommand{\kong}[1][0.5]{\vspace{#1cm}}

\begin{document}
	\author{ 路毅 \hspace{0.3cm} 曲阜师范大学 }
	\date{\number\year 年 \number\month 月 \number\day 日}
	\title{贝塞尔函数 柱函数}

\begin{frame}
	\maketitle
\end{frame}

\kaishu

\begin{frame}{贝塞尔函数 柱函数}
\begin{itemize}
	\item {\color{blue}第一节:贝塞尔函数的母函数与递推式}
	\vspace{1cm}
	\item 第二节:贝塞尔方程,贝塞尔函数作为正交函数系
	\vspace{1cm}
	\item 第三节:球贝塞尔函数
	\vspace{1cm}
	\item 第四节:第二类和第三类贝塞尔函数,渐进行为
\end{itemize}
\end{frame}

\begin{frame}{伽马函数}
为了后面叙述的方便,我们在这里简单介绍伽马函数。
\begin{equation}
\Gamma(x) = \int^\infty_0 e^{-t} t^{x-1} dt, x>0.
\end{equation}
容易求得
\begin{eqnarray}
\Gamma(1) &=& \int^\infty_0 e^{-t} dt = 1, \\
\Gamma(\frac{1}{2}) &=& \int^\infty_0 e^{-t} t^{-1/2} dt
= \int^\infty_0 e^{-\omega^2} 2 d \omega = \sqrt{\pi}.
\end{eqnarray}
根据定义式,做一次分部积分,可以推得$x>1$时的递推式
\begin{eqnarray}
\Gamma(x) &=& \int^\infty_0 e^{-t}t^{x-1} dt 
= - e^t t^{x-1} |^\infty_0 + (x-1)\int^\infty_0 e^{-t} t^{x-2} dt \nonumber\\
&=& (x-1) \Gamma(x-1).
\end{eqnarray}

\end{frame}

\begin{frame}{伽马函数}
所以,总结上面的结果,可以得到
\begin{eqnarray}
\Gamma(n+1) &=& n!, ~~~~ n = 0,1,2,\cdots, \\
\Gamma(n+\frac{1}{2}) &=& \frac{(2n)!}{2^{2n} n!} \sqrt{\pi}, ~~~~ n = 0,1,2,\cdots.
\end{eqnarray}
\end{frame}

\begin{frame}{贝塞尔函数的母函数}
贝塞尔函数的母函数为
\begin{equation}
G(x,z) = e^{\frac{x}{2}(z-\frac{1}{z})} = \sum^\infty_{n=-\infty} J_n(x) z^n,
\end{equation}
其中$J_n(x)$即贝塞尔函数, $x \in R$。因为母函数在$z=0$并不解析,但在$0<|z|<\infty$是解析的,所以可做洛朗展开,既有负幂次项,也有非负幂次项。

利用$e$指数函数的泰勒展开式,很容易写出$G(x,z)$的洛朗展开式,
\begin{equation}
G(x,z) = e^{\frac{xz}{2}} e^{-\frac{x}{2z}}
= \sum^\infty_{r=0} (\frac{x}{2})^r \frac{z^r}{r!}
\sum^\infty_{s=0} (-\frac{x}{2})^s \frac{z^{-s}}{s!},
\end{equation}
上式为两个级数相乘,进行重排,可以得到贝塞尔函数,$n \geq 0$时有
\begin{eqnarray}
J_n(x) &=& \sum^\infty_{s=0} \frac{(-1)^s}{s!(n+s)!}(\frac{x}{2})^{n+2s}, \\
J_{-n}(x) &=& \sum^\infty_{r=0}
\frac{(-1)^{n+r}}{r!(n+r)!}(\frac{x}{2})^{n+2r}
=\sum^\infty_{s=0} \frac{(-1)^s}{s!(-n+s)!}(\frac{x}{2})^{-n+2s}.
\end{eqnarray}
\end{frame}

\begin{frame}{贝塞尔函数}
可以看到$J_{-n}(x)$与$J_n(x)$的形式一致,且有
\begin{equation}
J_{-n}(x) = (-1)^n J_n(x).
\end{equation}
如果把阶乘形式推广到$\Gamma$函数,即$n! = \Gamma(n+1)$,则$J_n(x)$可以写作
\begin{equation}
J_n(x) = \sum^\infty_{s=0} \frac{(-1)^s}{s!\Gamma(n+s+1)} (\frac{x}{2})^{n+2s}.
\end{equation}
因为$\Gamma$函数在正实轴上都有定义,所以这个形式可以推广到任意实数阶贝塞尔函数$J_\nu$,
\begin{equation}
J_\nu(x) = \sum^\infty_{s=0} \frac{(-1)^s}{s!\Gamma(\nu+s+1)} (\frac{x}{2})^{\nu+2s}.
\end{equation}
当然,$G(x,z)$的洛朗展开只有整数阶的幂次,所以$J_\nu$只是$J_n$的推广,与$G(x,z)$没有直接关系。
\end{frame}

\begin{frame}{递推公式}
从母函数出发,可以推出$J_n$的递推公式,但是不能直接用到$J_\nu$,毕竟$J_\nu$与母函数没有直接关系。
所以,我们采用教材上的思路,从级数表达式出发,推导递推公式,因为$J_n, J_\nu$都有统一的级数表达式,所以这样推出的递推公式也是普适的。
因为
\begin{equation}
J_\nu(x) = \sum^\infty_{k=0} \frac{(-1)^k}{k!\Gamma(\nu+k+1)} (\frac{x}{2})^{\nu+2k}.
\end{equation}
所以有
\begin{eqnarray}
\frac{d}{dx}[ x^\nu J_\nu (x) ] &=& \frac{d}{dx}[ 2^\nu \sum^\infty_{k=0} \frac{(-1)^k}{k!\Gamma(\nu+k+1)} (\frac{x}{2})^{2\nu+2k} ] \nonumber\\
&=& 2^\nu \sum^\infty_{k=0} \frac{(-1)^k (k+\nu)}{k!\Gamma(\nu+k+1)} (\frac{x}{2})^{2\nu+2k-1}  \nonumber\\
&=& x^\nu \sum^\infty_{k=0} \frac{(-1)^k }{k!\Gamma(\nu+k)} (\frac{x}{2})^{\nu-1+2k}
= x^\nu J_{\nu-1}(x),
\end{eqnarray}

\end{frame}

\begin{frame}{递推公式}
类似地,可以证明
\begin{equation}
\frac{d}{dx}[ x^{-\nu} J_\nu(x) ] = - x^{-\nu} J_{\nu+1}(x).
\end{equation}
上面得到的两个式子即
\begin{eqnarray}
\nu J_\nu + xJ'_\nu &=& x J_{\nu-1}, \\
- \nu J_\nu + x J'_\nu &=& - x J_{\nu+1}.
\end{eqnarray}
分别消去$J'_\nu$和$J_\nu$,得到两个基本递推式
\begin{eqnarray}
J_{\nu-1} + J_{\nu+1} &=& \frac{2\nu}{x} J_\nu, 
\label{eqn:Bessel-recursion1}
\\
J_{\nu-1} - J_{\nu+1} &=& 2 J'_\nu.
\label{eqn:Bessel-resursion2}
\end{eqnarray}
注意特殊情况:
在$\nu = 0$时,有
\begin{equation}
J'_0 (x) = - J_1(x).
\end{equation}
\end{frame}

\begin{frame}{递推公式总结}

\begin{eqnarray}
\nu J_\nu + xJ'_\nu &=& x J_{\nu-1}, \\
- \nu J_\nu + x J'_\nu &=& - x J_{\nu+1}, \\
J_{\nu-1} + J_{\nu+1} &=& \frac{2\nu}{x} J_\nu, 
\\
J_{\nu-1} - J_{\nu+1} &=& 2 J'_\nu.
\end{eqnarray}

\end{frame}

\begin{frame}{半奇数阶贝塞尔函数}
利用定义,可以得到
\begin{eqnarray}
J_{\frac{1}{2}}(x) &=& (\frac{x}{2})^{\frac{1}{2}} \sum^\infty_{k=0} \frac{ (-1)^k }{ k! \Gamma(\frac{3}{2}+k) } (\frac{x}{2})^{2k}
 \nonumber\\
 &=& (\frac{x}{2})^{\frac{1}{2}} 
 \sum^\infty_{k=0} \frac{ (-1)^k 2^{2k+2} (k+1)! }{ k! (2k+2)! \sqrt{\pi} } (\frac{x}{2})^{2k}
 \nonumber\\
 &=& \sqrt{\frac{2}{\pi x}} \sum^\infty_{k=0} \frac{(-1)^k}{(2k+1)!}x^{2k+1} = \sqrt{\frac{2}{\pi x}} \sin x.
\end{eqnarray}
\begin{eqnarray}
J_{-\frac{1}{2}}(x) &=& \sum^\infty_{k=0} \frac{(-1)^k}{k! \Gamma(k+\frac{1}{2})} (\frac{x}{2})^{-\frac{1}{2}+2k}
 = \sum^\infty_{k=0} \frac{(-1)^k 2^{2k} k! }{k! (2k)! \sqrt{\pi} } (\frac{x}{2})^{-\frac{1}{2}+2k}
 \nonumber\\
 &=& \sqrt{\frac{2}{\pi x}} \sum^\infty_{k=0} \frac{(-1)^k}{(2k)!}x^{2k} = \sqrt{\frac{2}{\pi x}}\cos x.
\end{eqnarray}
结合前面的递推式,可以得知,所有半奇数阶的贝塞尔函数都可以写成初等函数形式。
\end{frame}


\begin{frame}{积分形式}
根据母函数
\begin{equation}
G(x,z) = e^{\frac{x}{2}(z-\frac{1}{z})} = \sum^\infty_{n=-\infty} J_n(x) z^n.
\end{equation}
以及洛朗展开公式
\begin{equation}
f(z) = \sum^\infty_{n=-\infty} c_n (z-a)^n, ~~
c_n = \frac{1}{2\pi i} \oint_\gamma \frac{f(\zeta)}{(\zeta-a)^{n+1}} d\zeta, n\in Z,
\end{equation}
$\gamma$为以$z=a$为圆心的收敛圆环内,且包围$z=a$点。

我们可以写出$J_n(x)$的柯西积分形式
\begin{equation}
J_n(x) = \frac{1}{2\pi i} \oint_C \frac{e^{\frac{x}{2}(\zeta-\frac{1}{\zeta})}}{\zeta^{n+1}} d\zeta,
\end{equation}
$C$为围绕$z=0$的任意曲线。

\end{frame}

\begin{frame}{积分形式}
不妨取$C$为单位圆,$\zeta = e^{i\theta}, \theta \in [-\pi, \pi]$,则有
\begin{eqnarray}
J_n(x) &=& \frac{1}{2\pi i} \int^\pi_{-\pi} e^{-i(n+1)\theta} e^{ix\sin\theta} ie^{i\theta} d\theta
\nonumber\\
&=& \frac{1}{2\pi} \int^\pi_{-\pi} e^{i(x\sin\theta - n\theta)} d \theta
\nonumber\\
&=& \frac{1}{2\pi} \int^\pi_{-\pi} \cos( x\sin\theta - n \theta ) d \theta.
\end{eqnarray}
更一般地,有
\begin{eqnarray}
J_\nu(x) &=& \frac{(\frac{x}{2})^\nu}{\Gamma(\frac{1}{2}) \Gamma(\nu+\frac{1}{2})}
\int^\pi_0 \cos(x\cos \theta) \sin^{2\nu} \theta d\theta, (Re \nu > - \frac{1}{2})
\nonumber\\
&=& \frac{1}{2\pi} \int^\pi_{-\pi} \cos( x \sin \theta - \nu \theta ) d\theta
-
\frac{\sin \nu \pi}{\pi} \int^\infty_0 e^{-x\sinh \zeta - \nu \zeta} d\zeta.
\nonumber\\
\end{eqnarray}
\end{frame}

\begin{frame}{贝塞尔函数 柱函数}
\begin{itemize}
	\item {第一节:贝塞尔函数的母函数与递推式}
	\vspace{1cm}
	\item {\color{blue}第二节:贝塞尔方程,贝塞尔函数作为正交函数系}
	\vspace{1cm}
	\item 第三节:球贝塞尔函数
	\vspace{1cm}
	\item 第四节:第二类和第三类贝塞尔函数,渐进行为
\end{itemize}
\end{frame}

\begin{frame}{贝塞尔方程}
利用上面推导的递推式,可以推导出贝塞尔方程,根据(\ref{eqn:Bessel-recursion1}-\ref{eqn:Bessel-resursion2}),有
\begin{eqnarray}
4 J''_\nu &=& 2J'_{\nu-1} - 2J'_{\nu+1}
= J_{\nu-2} - J_\nu - ( J_\nu - J_{\nu+2} )
\nonumber\\
&=& J_{\nu-2}(x) + J_{\nu+2} - 2 J_\nu
\nonumber\\
&=& \frac{2(\nu-1)}{x} J_{\nu-1} - J_\nu + \frac{2(\nu+1)}{x}J_{\nu+1} - J_\nu - 2 J_\nu
\nonumber\\
&=& \frac{2}{x}[n(J_{\nu-1} + J_{\nu+1}) + J_{\nu+1} - J_{\nu-1} ] - 4 J_\nu
\nonumber\\
&=& \frac{2}{x}[ \nu \frac{2\nu}{x} J_\nu - 2 J'_\nu ] - 4 J_\nu,
\end{eqnarray}
即
\begin{equation}
x^2 J''_\nu(x) + x J'_\nu(x) + (x^2 - \nu^2) J_\nu(x) = 0.
\end{equation}
这就是贝塞尔方程。
\end{frame}

\begin{frame}{正交性}
$\nu$阶贝塞尔方程可写作
\begin{equation}
x \frac{d^2}{dx^2}J_\nu(x) + \frac{d}{dx}J_\nu(x) + (x - \nu^2/x) J_\nu(x) = 0.
\end{equation}
做替换 $x \rightarrow kx, \frac{d}{dx} \rightarrow \frac{d}{d(kx)} = \frac{1}{k}\frac{d}{dx}$,则方程变为
\begin{equation}
kx \frac{1}{k^2}\frac{d^2}{dx^2} J_\nu(kx) + \frac{1}{k}\frac{d}{dx}J_\nu(kx) + (kx - \frac{\nu^2}{kx}) J_\nu(kx) = 0.
\end{equation}
即
\begin{equation}
x \frac{d^2}{dx^2}J_\nu(kx) + \frac{d}{dx}J_\nu(kx) + (k^2x - \nu^2/x) J_\nu(kx) = 0.
\end{equation}
\end{frame}

\begin{frame}{正交性}

\begin{equation}
x \frac{d^2}{dx^2}J_\nu(kx) + \frac{d}{dx}J_\nu(kx) + (k^2x - \nu^2/x) J_\nu(kx) = 0.
\end{equation}
也可以看做$Lu + \lambda \omega(x) u = 0$形式,
\begin{equation}
L = x \frac{d^2}{dx^2} + \frac{d}{dx} - \nu^2/x, ~~ \lambda = k^2, \omega(x) = x,
\end{equation}
取$k^\nu_i = \alpha_{\nu i} / a$,其中$\alpha_{\nu i}$为$J_\nu(x)$的第$i$个{\bf 正}零点,$i=1,2,\cdots$,即 $J_\nu (k^\nu_i a) = J_\nu (\alpha_{\nu i}) = 0$。
则有
\begin{equation}
x [ J_\nu(k^\nu_i x) \frac{d}{dx} J_\nu(k^\nu_j x) - \frac{d}{dx}J_\nu(k^\nu_i x) J_\nu(k^\nu_j x) ]|^a_0 = 0.
\end{equation}
根据斯图姆刘维尔理论,$i \neq j$时,有
\begin{equation}
\int^a_0 x J_\nu(k^\nu_i x) J_\nu(k^\nu_j x) dx = 0. 
\end{equation}
\end{frame}

\begin{frame}{$J_\nu(k^\nu_i x)$的模方}
我们可以另设$J_\nu(\xi x)$,$\xi$为$k_i$的小邻域内的实数,$\xi \in (k^\nu_i - \delta, k^\nu_i + \delta)$,根据贝塞尔方程有
\begin{eqnarray}
&& \frac{d}{dx}[x \frac{d}{dx} J_\nu(k^\nu_i x)] + [(k^\nu_i)^2 x - \frac{\nu^2}{x}] J_\nu(k^\nu_i x) = 0, \\
&& \frac{d}{dx}[x \frac{d}{dx} J_\nu(\xi x)] + (\xi^2 x - \frac{\nu^2}{x}) J_\nu(\xi x) = 0.
\end{eqnarray}
第一式乘以$J_\nu(\xi x)$,第二式乘以$J_\nu(k^\nu_i x)$,然后相减,在$[0,a]$上积分,得到
\begin{eqnarray}
&&((k^\nu_i)^2 - \xi^2) \int^a_0 x J_\nu(k^\nu_i x) J_\nu(\xi x) dx
\nonumber\\
&=& [ \xi x J_\nu(k^\nu_i x) \frac{d}{dx}J_\nu(\xi x) - k^\nu_i x \frac{d}{dx}J_\nu(k^\nu_i x) J_\nu(\xi x) ] |^a_0 
\nonumber\\
&=& - k^\nu_i a \frac{d}{dx}J_\nu( k^\nu_i a ) J_\nu( \xi a).
\end{eqnarray}
\end{frame}

\begin{frame}{$J_\nu(k^\nu_i x)$的模方}
所以,如果取极限$\xi \rightarrow k^\nu_i$,则有
\begin{eqnarray}
\int^a_0 x J^2_\nu(k^\nu_i x) dx
&=& \lim\limits_{\xi \rightarrow k^\nu_i} \int^a_0 x J_\nu(k^\nu_i x) J_\nu(\xi x) dx
\nonumber\\
&=& \lim\limits_{\xi \rightarrow k^\nu_i} \frac{- k^\nu_i aJ'_\nu(k^\nu_ia) J_\nu(\xi a)}{(k^\nu_i)^2 - \xi^2}
\nonumber\\
&=& \frac{ -k^\nu_i a^2 [ J'_\nu(k^\nu_i a )]^2 }{ - 2 k^\nu_i }
{\text{(洛必达法则)}}
\nonumber\\
&=& \frac{a^2}{2} [ J'_\nu(k^\nu_i a )]^2.
\end{eqnarray}

根据递推公式$ - \nu J_\nu + x J'_\nu = -x J_{\nu+1}$,以及$J_\nu(k^\nu_i a) = 0$,有
\begin{equation}
J'_\nu(k^\nu_i a) = - J_{\nu+1}(k^\nu_i a).
\end{equation}
所以,最终的最终,我们得到了$J_\nu(k^\nu_i x)$的模方
\begin{equation}
\int^a_0 x J^2_\nu( \alpha_{\nu n} x / a) dx = \frac{a^2}{2} [ J_{\nu+1}( \alpha_{\nu i}) ]^2.
\end{equation}
\end{frame}

\begin{frame}{正交归一性}

总结正交性、模方,得到:
\begin{equation}
\int^a_0 x J_\nu( k^\nu_i x ) J_\nu( k^\nu_j x ) dx = \delta_{ij} \frac{a^2}{2} [ J_{\nu+1}( \alpha_{\nu i}) ]^2.
\end{equation}

\end{frame}

\begin{frame}{展开定理的叙述(完备性未给出证明)}
设函数$f(r)$在区间$(0,a)$内有连续的一阶导数和分段连续的二阶导数,且$f(r)$在$r=0$处有界,在$r=a$处为零,则$f(r)$在$(0,a)$上可以展开为绝对且一致收敛的级数
\begin{eqnarray}
f(r) &=& \sum^\infty_{i=1} c_{\nu i} J_\nu ( k^\nu_i r), 
\end{eqnarray}
展开系数$c_{\nu i}$为
\begin{equation}
c_{\nu i} = \frac{2}{a^2 [J_{\nu +1}(\alpha_{\nu i})]^2}
\int^a_0 f(\rho) J_\nu( k^\nu_i \rho ) \rho d \rho.
\end{equation}
\end{frame}

\begin{frame}{圆膜振动问题}
固定边界的圆膜振动问题:
\begin{equation}
\left\{
\begin{aligned}
& u_{tt} = a^2( u_{xx} + u_{yy} ) ( 0 \leq x^2 + y^2 < l^2, t>0 ), \\
& u|_{x^2 + y^2 + l^2} = 0 ( t \geq 0), \\
& u(x,y,0) = \varphi(x,y), u_l(x,y,0) = \psi(x,y), ( 0 \leq x^2 + y^2 \leq l^2 ).
\end{aligned}
\right.
\end{equation}
做分离变量法,$u(x,y,t) = T(t) U(x,y)$,得到
\begin{equation}
\left\{
\begin{aligned}
& T'' + a^2 \lambda T = 0, \\
& U_{xx} + U_{yy} + \lambda U = 0, \\
& U|_{x^2 + y^2 = l^2} = 0,
\end{aligned}
\right.
\end{equation}
其中$\lambda$为待定常数。在$x-y$平面上取极坐标$(r,\phi)$,则方程变为
\begin{equation}
\left\{
\begin{aligned}
& T'' + a^2 \lambda T = 0, \\
& U_{rr} + \frac{1}{r}U_r + \frac{1}{r^2}U_{\phi \phi} + \lambda U = 0, \\
& U|_{r=l} = 0,
\end{aligned}
\right.
\end{equation}
要满足边界条件,我们会发现$\lambda = k^2, k \in R$。
\end{frame}

\begin{frame}{圆膜振动方程}
再令$U = \Phi(\phi) R(r)$,得到
\begin{equation}
\left\{
\begin{aligned}
& T'' + a^2 k^2 T = 0, \\
& \Phi'' + \nu^2 \Phi = 0, \\
& r^2 R'' + r R' + (k^2 r^2 - \nu^2 ) R = 0,\\
& R(l) = 0.
\end{aligned}
\right.
\end{equation}
就这样,贝塞尔方程出现了。
考虑周期性边界条件(或连续性条件),$\nu$必为整数$\nu = n$,所以$k = \alpha_{n,i}/l, i=1,2,...$时,$J_n(kr)$在$[0,l]$上构成正交归一完备函数基。即有
\begin{eqnarray}
u(r,\phi, t) &=& \sum^\infty_{n=0} \sum_{\infty}^{i=1}
[ ( A_{n,i} \cos a k^n_i t + B_{n,i} \sin a k^n_i t ) \cos n \phi 
\nonumber\\
&& + ( \alpha_{n,i} \cos a k^n_i t + \beta_{n,i} \sin a k^n_i t ) \sin n \phi ] J_n (k^n_i r).
\end{eqnarray}
\end{frame}

\begin{frame}{圆膜振动方程}
\begin{eqnarray}
u(r,\phi, t) &=& \sum^\infty_{n=0} \sum_{\infty}^{i=1}
[ ( A_{n,i} \cos a k^n_i t + B_{n,i} \sin a k^n_i t ) \cos n \phi 
\nonumber\\
&& + ( \alpha_{n,i} \cos a k^n_i t + \beta_{n,i} \sin a k^n_i t ) \sin n \phi ] J_n (k^n_i r).
\end{eqnarray}
代入初始条件,
\begin{eqnarray}
\varphi(r, \phi) &=& u(r,\phi,t=0) = \sum^\infty_{n=0} \sum^\infty_{i=1} ( A_{n,i} \cos n \phi + \alpha_{n,i} \sin n \phi ) J_n (k^n_i r),
\\
\psi(r, \phi) &=& u_t(r, \phi, t) |_{t=0}
= \sum^\infty_{n=0} \sum^\infty_{i=1}
ak^n_i ( B_{n,i} \cos n\phi + \beta_{n,i} \sin n \phi ) J_n(k^n_i r).
\nonumber\\
\end{eqnarray}
$\left\{ 1, \cos \phi, \cdots, \sin \phi, \sin 2\phi, \cdots \right\}$是$[0,2\pi]$上的正交基矢,而$ \left\{ J_n(k^n_1 r), J_n(k^n_2 r), \cdots \right\} $是$[0,l]$上的正交基矢,利用这些正交性,可以得到上式中的展开系数。
\end{frame}

\begin{frame}{诺依曼函数:第二类贝塞尔函数}
贝塞尔方程:
\begin{equation}
x^2 y'' + xy' + (x^2 - \nu^2) y = 0,
\end{equation}
在$\nu \notin Z $时,$J_\nu, J_{-\nu}$构成两个线性无关解,所以通解为$ C_1 J_\nu + C_2 J_{-\nu}$。

但是在$\nu \in Z$,即$\nu = n$时,有
\begin{equation}
J_{-n}(x) = (-1)^n J_n (x),
\end{equation}
所以$J_{-n}$与$J_n$线性相关,需要构造另一个线性无关解,才能写出$n$阶贝塞尔方程的通解。
\end{frame}

\begin{frame}{诺依曼函数:第二类贝塞尔函数}
定义诺依曼函数
\begin{equation}
Y_\nu (x) = \frac{ \cos \nu \pi J_\nu (x) - J_{-\nu}(x) }{ \sin \nu \pi},
\end{equation}
$\nu$为整数时,分子分母都趋于0,但整体是个有限大的数
\begin{eqnarray}
Y_n(x) &=& \lim\limits_{\nu \rightarrow n}
\frac{ -\pi \sin \nu \pi J_\nu + \cos \nu \pi \frac{\partial J_\nu}{\partial \nu} - \frac{\partial J_{-\nu}}{\partial \nu} }{ \pi \cos \nu \pi }
\nonumber\\
&=& \frac{1}{\pi}[ \frac{\partial J_\nu}{\partial \nu} - (-1)^n \frac{ \partial J_{-\nu}}{ \partial \nu } ] |_{\nu = n}
\end{eqnarray}
显然,$Y_{-n}(x) = (-1)^n Y_n(x)$。
\end{frame}

\begin{frame}{$Y_n$是$n$阶贝塞尔方程的解}

因为$ x^2 J''_\nu + x J'_\nu + (x^2 - \nu^2) J_\nu = 0$,所以对这个公式两边求关于$\nu$的偏导数,得到
\begin{equation}
[ x^2 (\frac{\partial J_\nu}{\partial \nu})'' + x (\frac{\partial J_\nu}{\partial \nu})'' + (x^2 - \nu^2)(\frac{\partial J_\nu}{\partial \nu}) ]_{\nu = n} = 2n J_n,
\end{equation}
类似地,对$J_{-\nu}$做相似的事,得到
\begin{equation}
[ x^2 (\frac{\partial J_{-\nu} }{\partial \nu})'' + x (\frac{\partial J_{-\nu} }{\partial \nu})'' + (x^2 - \nu^2)(\frac{\partial J_{-\nu}}{\partial \nu}) ]_{\nu = n} = 2n J_{-n} = 2n (-1)^n J_n(x),
\end{equation}
利用上面两个式子,以及$Y_n(x) = \frac{1}{\pi}[ \frac{\partial J_\nu}{\partial \nu} - (-1)^n \frac{ \partial J_{-\nu}}{ \partial \nu } ] |_{\nu = n}$,得到
\begin{equation}
[ x^2 Y''_\nu + x Y'_\nu + (x^2 - \nu^2) Y_\nu ]|_{\nu = n} = 0,
\end{equation}
即$Y_n(x)$满足Bessel方程,与$J_n(x)$构成$n$阶Bessel方程的两个线性无关解。
\end{frame}

\begin{frame}{汉克尔函数}
\begin{eqnarray}
H^{(1)}_\nu(x) &=& J_\nu(x) + i Y_\nu(x) = \frac{1}{i\sin \nu \pi}[ J_{-\nu}(x) - e^{-i\nu\pi}J_\nu(x) ] , \\
H^{(2)}_\nu(x) &=& J_\nu(x) - i Y_\nu(x) = \frac{1}{i\sin \nu \pi}[ e^{i\nu\pi} J_\nu(x) - J_{-\nu}(x) ].
\end{eqnarray}
\end{frame}

\begin{frame}{球贝塞尔函数}
在三维球坐标下,分解一些物理方程时,可能遇到
\begin{equation}
r^2 R'' + 2r R' + [k^2 r^2 - \nu(\nu+1)] R = 0,
\end{equation}
可以设$R(r) = x^{-1/2} z(x)$,则方程转化为
\begin{equation}
x^2 z'' + x z' + [x^2 - (\nu+\frac{1}{2})^2] z = 0,
\end{equation}
正是$\nu+\frac{1}{2}$阶Bessel方程,所以两个线性无关解为
\begin{equation}
\left\{
\begin{aligned}
& x^{-\frac{1}{2}} J_{\nu+\frac{1}{2}}(x), \\
& x^{-\frac{1}{2}} Y_{\nu+\frac{1}{2}}(x),
\end{aligned}
\right.
\text{或者}
\left\{
\begin{aligned}
& x^{-\frac{1}{2}} H^{(1)}_{\nu+\frac{1}{2}}(x), \\
& x^{-\frac{1}{2}} H^{(2)}_{\nu+\frac{1}{2}}(x).
\end{aligned}
\right.
\end{equation}
\end{frame}

\begin{frame}{球Bessel方程}
定义球贝塞尔函数
\begin{eqnarray}
j_\nu(x) &=& \sqrt{ \frac{\pi}{2x} } J_{\nu+\frac{1}{2}}(x), \\
n_\nu(x) &=& \sqrt{ \frac{\pi}{2x} } Y_{\nu+\frac{1}{2}}(x), \\
h^{(1)}_\nu(x) &=& \sqrt{ \frac{\pi}{2x} } H^{(1)}_{\nu+\frac{1}{2}}(x), \\
h^{(1)}_\nu(x) &=& \sqrt{ \frac{\pi}{2x} } H^{(1)}_{\nu+\frac{1}{2}}(x).
\end{eqnarray}
显然有
\begin{eqnarray}
h^{(1)}_\nu(x) &=& j_\nu(x) + i n_\nu(x), \\
h^{(2)}_\nu(x) &=& j_\nu(x) - i n_\nu(x), \\
j_0(x) &=& \frac{\sin x}{x}, j_{-1}(x) = \frac{\cos x}{x}.
\end{eqnarray}
\end{frame}

\begin{frame}{虚宗量Bessel函数}
$\nu$阶Bessel方程:
\begin{equation}
x^2 y'' + x y' + (x^2 - \nu^2) y = 0,
\end{equation}
做替换$x \rightarrow ix$,得到
\begin{equation}
x^2 y'' + x y' - (x^2 + \nu^2) y = 0,
\end{equation}
其解为$J_\nu(ix)$,记
\begin{equation}
I_\nu(x) = i^{-\nu} J_\nu(ix),
\end{equation}
称作第一类虚宗量Bessel函数。
上面的偏微分方程的另一个线性无关解为
\begin{equation}
K_\nu(x) = \frac{\pi}{2} \frac{I_{-\nu}(x) - I_\nu(x)}{\sin \nu \pi},
\end{equation}
称作第二类虚宗量Bessel函数。
可以证明,
\begin{equation}
K_\nu(x) = \frac{\pi i^{\nu+1}}{2} H^{(1)}_\nu(ix).
\end{equation}
\end{frame}

\begin{frame}{Bessel函数的渐进公式}
$|x| \rightarrow \infty$时,有
\begin{eqnarray}
J_\nu(x) & \rightarrow & \frac{2}{\pi x} \cos( x - \frac{\nu\pi}{2} - \frac{\pi}{4}), \\
Y_\nu(x) & \rightarrow & \frac{2}{\pi x} \sin( x - \frac{\nu\pi}{2} - \frac{\pi}{4}), \\
H^{(1)}_\nu(x) & \rightarrow & \frac{2}{\pi x} e^{i( x - \frac{\nu\pi}{2} - \frac{\pi}{4})}, \\
H^{(2)}_\nu(x) & \rightarrow & \frac{2}{\pi x} e^{-i( x - \frac{\nu\pi}{2} - \frac{\pi}{4})}, \\
I_\nu(x) & \rightarrow & \frac{1}{\sqrt{2\pi x}} e^x, \\
K_\nu(x) & \rightarrow & \frac{1}{\sqrt{2\pi x}} e^{-x}.
\end{eqnarray}
\end{frame}

\begin{frame}{课后作业}

习题 1, 2, 3, 8, 11

\end{frame}

\end{document}

\begin{frame}{贝塞尔函数的渐进行为}
$\nu$阶Bessel方程:
\begin{equation}
x^2 y'' + x y' + (x^2 - \nu^2) y = 0,
\end{equation}
设$y = x^{-\frac{1}{2}}z$,则可以推得
\begin{equation}
z'' + (1+\frac{1-4\nu^2}{4x^2}) z = 0,
\end{equation}
$|x| \rightarrow \infty$时,方程趋于
\begin{equation}
z'' + z = 0,
\end{equation}
即$z$趋于正弦余弦形式,
\end{frame}