\documentclass[11pt]{beamer}
\usepackage[UTF8,scheme=plain]{ctex}
\usepackage{listings}
\usepackage[utf8]{inputenc}
\usepackage[T1]{fontenc}
\usepackage{amsmath}
\usepackage{amsfonts}
\usepackage{amssymb}
\usepackage{graphicx}
\usetheme{Boadilla}

\usepackage{framed} % 可以用 \begin{shaded},即背景色块
\definecolor{shadecolor}{rgb}{0.9,0.9,0.9}

\newcommand{\kong}[1][0.5]{\vspace{#1cm}}

\begin{document}
	\author{ 路毅 \hspace{0.3cm} 曲阜师范大学 }
	\date{\number\year 年 \number\month 月 \number\day 日}
	\title{数学物理方法第九章}

\begin{frame}
	\maketitle
\end{frame}

\kaishu

\begin{frame}{第九章:拉普拉斯方程的圆的狄利克雷问题的傅里叶解}
\begin{itemize}
	\item 第一节:圆的狄利克雷问题
	\vspace{1cm}
	\item 第二节:$\delta$函数
\end{itemize}
\end{frame}

\begin{frame}{拉普拉斯方程}
拉普拉斯方程
\begin{equation}
u_{xx} + u_{yy} + u_{zz} = 0,
\end{equation}
或写作
\begin{equation}
\Delta u = \nabla^2 u = 0,
\end{equation}
其中$\vec{\nabla} = \vec{e}_x \partial_x + \vec{e}_y \partial_y + \vec{e}_z \partial_z$为梯度算符。

\kong[1]
真空静电势、稳态无源处温度分布、扩散过程稳态无源处物质浓度,都满足这个方程。
对应的物理定律分别为:无电荷处静电场散度为0、能量守恒、物质不灭。

\end{frame}

\begin{frame}{无限长圆柱稳态温度分布}
无限长圆柱以$z$轴为对称轴,表面温度$u(l,\theta,z)=u(l,\theta)=f(\theta)$(l为圆柱半径),与$z$无关,即沿$z$轴具有平移对称性。
足够长时间后,圆柱达到热平衡,各个横截面温度分布完全相同,故各处温度可记为$u(r,\theta)$,与$z$无关。
$u$在柱体内部满足拉普拉斯方程
\begin{equation}
u_{xx} + u_{yy} = 0,
\end{equation}
利用二维平面梯度算符
\begin{eqnarray}
\vec{\nabla} &=& \vec{e}_x \partial_x + \vec{e}_y \partial_y = \vec{e}_r \partial_r + \vec{e}_\theta \frac{1}{r}\partial_\theta, \\
\partial_r \vec{e}_r &=& \partial_r \vec{e}_\theta = 0, \\
\partial_\theta \vec{e}_r &=& \vec{e}_\theta, ~~ \partial_\theta \vec{e}_\theta = - \vec{e}_r,
\end{eqnarray}
可以得到
\begin{equation}
\nabla^2 = \partial_{xx} + \partial_{yy} = \partial_{rr} + \frac{1}{r^2}\partial_{\theta \theta} + \frac{1}{r} \partial_r.
\end{equation}
\end{frame}

\begin{frame}{无限长圆柱稳态温度分布:圆的狄利克雷问题}

即得2维极坐标下的拉普拉斯方程
\begin{equation}
u_{rr} + \frac{1}{r}u_r + \frac{1}{r^2}u_{\theta \theta} = 0, 0\leq r \leq l, 0 \leq \theta \leq 2\pi.
\end{equation}
另有边界条件
\begin{equation}
u(l,\theta) = f(\theta).
\end{equation}
这就叫做“圆的狄利克雷问题”
\end{frame}

\begin{frame}{分离变量法}
故技重施,先设特殊形式
\begin{equation}
u(r,\theta) = \Theta(\theta)R(r),
\end{equation}
代入2维极坐标拉普拉斯方程,得到
\begin{equation}
r^2 \Theta(\theta) R''(r) + r \Theta(\theta)R'(r) + \Theta''(\theta)R(r) = 0,
\end{equation}
即
\begin{equation}
\frac{\Theta''(\theta)}{\Theta(\theta)} = - \frac{r^2 R''(r) + rR'(r)}{R(r)} = - \lambda,
\end{equation}
即
\begin{equation}
\left\{
\begin{aligned}
&& \Theta''(\theta) + \lambda \Theta(\theta) = 0, \\
&& r^2 R''(r) + rR'(r) - \lambda R(r) = 0.
\end{aligned}
\right.
\end{equation}
由于$u(r,\theta+2\pi) = u(r,\theta)$,有边界条件
\begin{equation}
\Theta(\theta+2\pi) = \Theta(\theta), 
\end{equation}
\end{frame}

\begin{frame}{分离变量法}
于是得到
\begin{equation}
\Theta(\theta) = a_n \cos(n\theta) + b_n \sin(n\theta), n=0,1,\cdots 
\end{equation}
$R(r)$的通解为
\begin{eqnarray}
R_0(r) &=& c_0 + d_0 \ln r, n=0, \\
R_n(r) &=& c_n r^n + d_n r^{-n}, n=1,2,\cdots 
\end{eqnarray}
因为$u(0,\theta)$是有限大的值,所以$R(0)=finite$,于是有
\begin{eqnarray}
R_0(r) &=& c_0 , n=0, \\
R_n(r) &=& c_n r^n, n=1,2,\cdots 
\end{eqnarray}
所以$u_n(r,\theta) = R_n(r) \Theta_n(\theta)$可以写作
\begin{equation}
u_n(r,\theta) = \left\{
\begin{aligned}
& \frac{A_0}{2}, & n=0 \\
& r^n (A_n \cos(n\theta) + B_n \sin(n\theta) ), & n=1,2,\cdots
\end{aligned}
\right.
\end{equation}

\end{frame}

\begin{frame}{分离变量法}
所有$u_n$叠加起来得到
\begin{equation}
u(r,\theta) = \sum^\infty_{n=0} u_n(r,\theta),
\end{equation}
它满足$u_{rr} + \frac{1}{r}u_r + \frac{1}{r^2}u_{\theta\theta} = 0$,且有无限自由参数。
带入边界条件$u(l,\theta) = f(\theta)$,即得
\begin{equation}
\frac{A_0}{2} + \sum^\infty_{n=1}(A_n\cos(n\theta) + B_n\sin(n\theta) ) = f(\theta),
\end{equation}
所以$An,B_n$即傅里叶系数,
\begin{equation}
\left\{
\begin{aligned}
A_n &=& \frac{1}{\pi l^n} \int^{2\pi}_0 f(\varphi) \cos(n\varphi) d\varphi, n=0,1,\cdots \\
B_n &=& \frac{1}{\pi l^n} \int^{2\pi}_0 f(\varphi) \sin(n\varphi) d\varphi, n=1,2,\cdots
\end{aligned}
\right.
\end{equation}
\end{frame}

\begin{frame}{分离变量法}

代回$u(r,\theta)$的表达式,得到
\begin{equation}
u(r,\theta) = \frac{1}{2\pi} \int^{2\pi}_0 f(\varphi) [ 1+ 2\sum^\infty_{n=1} (r/l)^n \cos(\varphi-\theta) ] d\varphi,
\end{equation}
记$\delta = \varphi - \theta, x = \frac{r}{l} e^{i\delta}, y= \frac{r}{l} e^{-i\delta}$,在柱体内部,有$r<l$,即$|x|, |y|<1$,
\begin{eqnarray}
&& 1+ 2\sum^\infty_{n=1} (r/l)^n \cos(\varphi-\theta) \nonumber\\
&=& 1 + \sum^\infty_{n=1}(x^n + y^n) 
= \frac{1}{1-x} + \frac{1}{1-y} -1 = \frac{1-xy}{(1-x)(1-y)} \nonumber\\
&=& \frac{1-r^2/l^2}{1-2r\cos\delta/l + r^2/l^2}
= \frac{l^2 - r^2 }{l^2 - 2lr\cos(\varphi - \theta) + r^2}
\end{eqnarray}
所以我们得到
\begin{equation}
u(r,\theta) = \frac{1}{2\pi}\int^{2\pi}_0 f(\varphi) \frac{l^2 - r^2 }{l^2 - 2lr\cos(\varphi - \theta) + r^2} d \varphi.
\end{equation}
\end{frame}

\begin{frame}{物理阐释?}
\begin{equation}
u(r,\theta) = \frac{1}{2\pi}\int^{2\pi}_0 f(\varphi) \frac{l^2 - r^2 }{l^2 - 2lr\cos(\varphi - \theta) + r^2} d \varphi.
\end{equation}

\kong[1]
分母下面是$(l,\varphi,z)$与$(r,\theta,z)$的距离平方。
如此简洁的形式,一定存在一个简洁的解释,但我还没有想到。
\end{frame}

\begin{frame}{$\delta$函数}
狄拉克:$\delta$函数的定义为
\begin{eqnarray}
&&\delta(x) = \left\{
\begin{aligned}
&0      &(x\neq 0) \\
&\infty &(x=0)
\end{aligned}
\right. \\
&& \int^\infty_{-\infty} \delta(x) dx = 1.
\end{eqnarray}
容易证明
\begin{equation}
\int^\infty_{-\infty} \varphi(x) \delta(x) dx
= \int^\epsilon_{-\epsilon} \varphi(x) \delta(x) dx
= \varphi(\xi) \int^\epsilon_{-\epsilon} \delta(x) dx = \varphi(\xi),
\end{equation}
其中$\xi$为$-\epsilon,\epsilon$之间的一点,上式使用了积分中值定理。
$\epsilon \rightarrow 0$时,$\xi \rightarrow 0$,即得
\begin{equation}
\int^\infty_{-\infty} \varphi(x) \delta(x) dx = \varphi(0).
\end{equation}

\end{frame}

\begin{frame}{$\delta$函数平移}
\begin{eqnarray}
&&\delta(x - \xi) = \left\{
\begin{aligned}
&0      &(x\neq \xi) \\
&\infty &(x=\xi)
\end{aligned}
\right. \\
&& \int^\infty_{-\infty} \delta(x-\xi) dx = 1.
\end{eqnarray}
易得
\begin{equation}
\int^\infty_{-\infty} \varphi(x) \delta(x-\xi) dx = \varphi(\xi).
\end{equation}
\end{frame}

\begin{frame}{高维$\delta$函数}
高维$\delta$函数定义为
\begin{equation}
\delta(x,y,z) = \delta(x) \delta(y) \delta(z),
\end{equation}
则有
\begin{equation}
\int \int \int \varphi(x,y,z) \delta(x,y,z) dx dy dz = \varphi(0,0,0).
\end{equation}
\end{frame}

\begin{frame}{$\delta$函数性质}
i) $\delta$函数是偶函数

\kong[1]
ii) $x\delta(x) = 0$

\kong[1]
iii) $\delta(x) = \frac{d H(x)}{dx}$,其中$H(x)$称作单位阶跃函数,定义为
\begin{equation}
H(x) = \left\{
\begin{aligned}
&0, x<0, \\
&1, x>0.
\end{aligned}
\right.
\end{equation}

\end{frame}

\begin{frame}{作业}

\kong[1]

习题 2,3,7

\kong[1]

\end{frame}

\end{document}