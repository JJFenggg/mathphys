\documentclass[11pt]{beamer}
\usepackage[UTF8,scheme=plain]{ctex}
\usepackage{listings}
\usepackage[utf8]{inputenc}
\usepackage[T1]{fontenc}
\usepackage{amsmath}
\usepackage{amsfonts}
\usepackage{amssymb}
\usepackage{mathrsfs}
\usepackage{graphicx}
\usetheme{Boadilla}

\usepackage{framed} % 可以用 \begin{shaded},即背景色块
\definecolor{shadecolor}{rgb}{0.9,0.9,0.9}

\newcommand{\kong}[1][0.5]{\vspace{#1cm}}

\begin{document}
	\author{ 路毅 \hspace{0.3cm} 曲阜师范大学 }
	\date{\number\year 年 \number\month 月 \number\day 日}
	\title{数学物理方法第十三章}

\begin{frame}
	\maketitle
\end{frame}

\kaishu

\begin{frame}{第十三章:拉普拉斯变换}
\begin{itemize}
	\item 第一节:拉普拉斯变换的定义和它的逆变换
	\vspace{1cm}
	\item 第二节:拉普拉斯变换的性质和应用
\end{itemize}
\end{frame}

\begin{frame}{拉普拉斯变换的定义}
若函数$f(t)$满足条件(A):
\begin{itemize}
\item [i] 当$t<0$时$f(t)=0$
\item [ii] 当$t\geq 0$时,$f(t)$及$f'(t)$除去有限个第一类间断点以外处处连续。
\item [iii] 当$t \rightarrow \infty$时,$f(t)$的增长速度不超过某个指数函数,即存在常数$M$及$\sigma_0 \geq 0$,使得
\begin{equation}
|f(t)| \leq M e^{\sigma_0 t}, 0<t<\infty,
\end{equation}
其中$\sigma_0$称为$f(t)$的增长指数。
\end{itemize}
这时,我们称
\begin{equation}
L(p) = \mathscr{L}[f(t)] = \int^\infty_0 f(t) e^{-pt} dt,
\end{equation}
为函数$f(t)$的拉普拉斯变换(拉氏变换),其中$Re(p) > \sigma_0$。而称
\begin{equation}
f(t) = \mathscr{L}^{-1} [ L(p) ] = \frac{1}{2\pi i} \int^{\sigma + i \infty}_{\sigma - i \infty} L(p) e^{pt} dp
\end{equation}
为$L(p)$的拉普拉斯逆变换,其中$\sigma > \sigma_0$。
\end{frame}

\begin{frame}{例1}
例1:拉普拉斯变换$\mathscr{L}[ 1 ] = ?$
\begin{equation}
\mathscr{L}[1] = \int^\infty_0 1 e^{-pt} dt = - \frac{1}{p} e^{-pt} |^\infty_0 = \frac{1}{p}.
\end{equation}
注:教材与课件都约定:若$t <0$,原像函数值都为0。
%拉普拉斯逆变换$\mathscr{L}^{-1}[ \frac{1}{p} ] = ?$
%\begin{equation}
%\mathscr{L}^{-1}[ \frac{1}{p} ] = \frac{1}{2\pi i} %\int^{\sigma+i\infty}_{\sigma-i\infty} \frac{1}{p} e^{pt} dp = 
%\end{equation}
\end{frame}

\begin{frame}{拉普拉斯变换的性质}

\begin{itemize}
\item [1] 它是一个线性变换
\begin{equation}
\mathscr{L} [ \alpha f_1 + \beta f_2 ] = \alpha \mathscr{L}[f_1] + \beta \mathscr{L}[ f_2 ],
\end{equation}

\item [2] 乘积定理
\begin{equation}
\mathscr{L}[f_1(t) f_2(t) ] = \frac{1}{2\pi i} \int^{\sigma + i\infty}_{\sigma - i\infty} L_1(q) L_2(p-q) dq,
\end{equation}
其中$\sigma > \sigma_1$,$Re(p) > \sigma_2 + \sigma$。

\item [3] 原像的导数定理
\begin{eqnarray}
\mathscr{L}[f'(t)] &=& p \mathscr{L}[f(t)] - f(0) = p L(p) - f(0),
\nonumber\\
\mathscr{L}[f^{(n)}(t)] &=& p^n \mathscr{L}[f(t)] - p^{n-1} f(0) - p^{n-2} f'(0) - \cdots - f^{(n-1)}(0). \nonumber
\end{eqnarray}
若$f(0) = f'(0) = \cdots = f^{(n-1)}(0) = 0$,则有
\begin{equation}
\mathscr{L}[ f^{(n)}(t) ] = p^n \mathscr{L}[ f(t) ] = p^n L(p).
\end{equation}
\end{itemize}
\end{frame}

\begin{frame}{拉普拉斯变换的性质}
\begin{itemize}
\item [4] 原像的积分定理
\begin{equation}
\mathscr{L}[ \int^t_0 f(t)dt ] = \frac{ \mathscr{L}[f(t)]}{p} = \frac{L(p)}{p}.
\end{equation}
\item [5] 像的导数定理
\begin{equation}
L^{(n)}(p) = \mathscr{L}[ (-t)^n f(t) ].
\end{equation}


\end{itemize}

\end{frame}

\begin{frame}{例2}

由像的导数定理
\begin{equation}
\mathscr{L}[t] = - \mathscr{L}[ -t * 1 ] = - \frac{\mathscr{L}[1]}{dp} = \frac{1}{p^2}.
\end{equation}
继续下去,得到
\begin{equation}
\mathscr{L}[t^n] = \frac{n!}{p^{n+1}}, n=0,1,2,\cdots.
\end{equation}

\end{frame}

\begin{frame}{拉普拉斯变换的性质6-8}
\begin{itemize}
\item [6] 像的积分定理
\begin{equation}
\int^\infty_p L(p) dp = \mathscr{L}[ f(t)/t ].
\end{equation}

\item [7] 相似定理
设$a>0$,则有
\begin{equation}
\mathscr{L}[ f(at) ] = \frac{1}{a} L(p/a).
\end{equation}

\item [8] 位移定理:$L(p-p_0) = \mathscr{L}[ e^{p_0 t} f(t) ]$

\end{itemize}

\end{frame}

\begin{frame}{例3,4}
例3:利用位移定理,由例2得
\begin{equation}
\mathscr{L}[ e^{ \alpha t} ] = \mathscr{L}[ e^{ \alpha t} 1 ] = \frac{1}{p - \alpha}.
\end{equation}

例4: 利用前例结果,得到
\begin{eqnarray}
\mathscr{L}[ sin \omega t] &=& \mathscr{L}[ \frac{e^{i\omega t} - e^{-i\omega t} }{2i} ]
\nonumber\\ 
&=& \frac{1}{2i}\left\{ \mathscr{L}[e^{i\omega t}] - \mathscr{L}[e^{-i\omega t} \right\}
\nonumber\\
&=& \frac{1}{2i}[ \frac{1}{p-i\omega}  - \frac{1}{p+i\omega} ] = \frac{\omega}{p^2+\omega^2}.
\end{eqnarray}

\end{frame}

\begin{frame}{例5}
利用原像的导数定理,由例4有
\begin{equation}
\mathscr{L}[ \frac{d}{dt}\sin \omega t ] = p \mathscr{L}[ \sin \omega t ] = \frac{p \omega}{p^2 + \omega^2},
\end{equation}
另外有
\begin{equation}
\mathscr{L}[ \frac{d}{dt} \sin \omega t ] = \mathscr{L}[ \omega \cos \omega t ]
= \omega \mathscr{L}[ \cos \omega t ], 
\end{equation}
得到
\begin{equation}
\mathscr{L}[ \cos \omega t ] = \frac{p}{p^2 + \omega^2}.
\end{equation}
\end{frame}

\begin{frame}{拉普拉斯变换的性质9-10}
\begin{itemize}
\item[9] 滞后定理:设$\tau>0$,则
\begin{equation}
\mathscr{L}[ f(t-\tau) ] = e^{-p\tau} L(p).
\end{equation}

\item[10] 卷积定理 

定义卷积:若$f_1(t), f_2(t)$都满足条件(A),则称积分
\begin{equation}
\int^t_0 f_1(\tau) f_2(t-\tau) d\tau,
\end{equation}
为$f_1(t)$和$f_2(t)$的卷积,记为$f_1(t) * f_2(t)$,显然这个卷积函数也满足条件(A),且有
\begin{equation}
f_1(t) * f_2(t) = f_2(t) * f_1(t)
\end{equation}
卷积定理为
\begin{equation}
\mathscr{L}[ f_1(t)*f_2(t) ] = \mathscr{L}[f_1(t)] \mathscr{L}[f_2(t)].
\end{equation}
\end{itemize}
\end{frame}

\begin{frame}{例6}
单位阶跃函数
\begin{equation}
H(t-a) = \left\{
\begin{aligned}
& 0  & (t<a) \\
& 1  & (t>a).
\end{aligned}
\right.
\end{equation}
由滞后定理,可得
\begin{equation}
\mathscr{L}[ H(t-a) ] = \frac{ e^{-ap} }{p}.
\end{equation}
\end{frame}

\begin{frame}{例7}
$\delta(t)$函数的拉普拉斯变换
\begin{equation}
\mathscr{L}[ \delta(t) ] = 1,
\end{equation}
由滞后定理
\begin{equation}
\mathscr{L}[ \delta(t-\tau) ] = e^{-p\tau}.
\end{equation}
\end{frame}

\begin{frame}{例8}
求解初值问题
\begin{equation}
\left\{
\begin{aligned}
& x'(t) - x(t) = 1, \\
& x(0) = 0.
\end{aligned}
\right.
\end{equation}

\end{frame}

\begin{frame}{作业}


\kong[0.5]
课后习题:1,4,5

\kong[1]
\end{frame}


\begin{frame}{大作业}

\kong[0.5]
选取《数学物理方法》中任一章节内容,写一份调研小论文。

\kong[0.5]
示例:
\begin{itemize}
\item 留数定理的历史调研
\item 弦振动方程的达朗贝尔解
\item 格林函数与Wronskian方法
\item 傅里叶变换在音乐中的应用
\item 狄利克雷问题的适定性
\item ...
\end{itemize}

截止时间:6月15日24:00。

\kong[0.5]

重要的事说三遍:抄袭0分,抄袭0分,抄袭0分!

\end{frame}

\end{document}

