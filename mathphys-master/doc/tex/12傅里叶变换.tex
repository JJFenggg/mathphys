\documentclass[11pt]{beamer}
\usepackage[UTF8,scheme=plain]{ctex}
\usepackage{listings}
\usepackage[utf8]{inputenc}
\usepackage[T1]{fontenc}
\usepackage{amsmath}
\usepackage{amsfonts}
\usepackage{amssymb}
\usepackage{mathrsfs}
\usepackage{graphicx}
\usetheme{Boadilla}

\usepackage{framed} % 可以用 \begin{shaded},即背景色块
\definecolor{shadecolor}{rgb}{0.9,0.9,0.9}

\newcommand{\kong}[1][0.5]{\vspace{#1cm}}

\begin{document}
	\author{ 路毅 \hspace{0.3cm} 曲阜师范大学 }
	\date{\number\year 年 \number\month 月 \number\day 日}
	\title{数学物理方法第十二章}

\begin{frame}
	\maketitle
\end{frame}

\kaishu

\begin{frame}{第十二章:傅里叶变换}
\begin{itemize}
	\item 第一节:傅里叶变换的定义及其基本性质
	\vspace{1cm}
	\item 第二节:用傅里叶变换解微分方程举例
\end{itemize}
\end{frame}

\begin{frame}{傅里叶变换的定义}
在第八章,求解无限长杆热传导初值问题时,引入了傅里叶积分
\begin{equation}
f(x) = \frac{1}{2\pi} \int^\infty_{-\infty} d \lambda \int^\infty_{-\infty} f(\xi) \cos \lambda(x-\xi) d\xi,
\end{equation}
由于$\sin \lambda (x - \xi)$是$\lambda$的奇函数,所以有
\begin{equation}
f(x) = \frac{1}{2\pi} \int^\infty_{-\infty} d \lambda \int^\infty_{-\infty} f(\xi) e^{ i \lambda(x-\xi) } d\xi,
\end{equation}
记
\begin{equation}
F(\lambda) = \int^\infty_{-\infty} f(\xi) e^{-i\lambda \xi} d\xi,
\end{equation}
叫做$f(x)$的像,也记作 $ F(\lambda) = \mathscr{F}(f)$,$f(x)$称作$F(\lambda)$的原像,记作$f(x) = \mathscr{F}^{-1}(F(\lambda))$,逆变换$\mathscr{F}^{-1}$为
\begin{equation}
f(x) = \frac{1}{2\pi}\int^\infty_{-\infty} F(\lambda) e^{i\lambda x} d \lambda.
\end{equation}
\end{frame}


\begin{frame}{收敛性}
可以证明,若$f(x)$在$(-\infty,\infty)$绝对可积,在任一有限区间上最多有有限个极值点和第一类间断点,则$f(x)$的傅里叶变换$F(\lambda)$存在,且逆变换
\begin{equation}
\mathscr{F}^{-1}[F] = \frac{f(x+0)+f(x-0)}{2}.
\end{equation}

\end{frame}

\begin{frame}{傅里叶变换的性质}

\begin{itemize}
\item [1] 它是一个线性变换
\begin{equation}
\mathscr{F} [ \alpha f_1 + \beta f_2 ] = \alpha \mathscr{F}[f_1] + \beta \mathscr{F}[ f_2 ],
\end{equation}

\item [2] 卷积定理
定义卷积
\begin{equation}
f_1(x) \star f_2(x) = \int^\infty_{-\infty} f_1(x-\eta) f_2(\eta) d \eta,
\end{equation}
显然$f_1(x) \star f_2(x) = f_2(x) \star f_1(x)$。卷积定理为
\begin{equation}
\mathscr{F}(f_1 \star f_2) = \mathscr{F}[f_1] \mathscr{F}[f_2].
\end{equation}

\item [3] 乘积定理
\begin{equation}
\mathscr{F}[f_1 f_2] = \frac{1}{2\pi}\mathscr{F}[f_1] \star \mathscr{F}[f_2].
\end{equation}

\end{itemize}

\end{frame}

\begin{frame}{傅里叶变换的性质}
\begin{itemize}
\item [4] 原像的导数定理
若$|x| \rightarrow \infty$时,$f(x) \rightarrow 0$,则有
\begin{equation}
\mathscr{F}[f'] = i \lambda \mathscr{F}[f],
\end{equation}

\item [5] 像的导数定理
\begin{equation}
\frac{d}{d \lambda} \mathscr{F}[f] = \mathscr{F}[ -i x f].
\end{equation}
\end{itemize}
\end{frame}

\begin{frame}{$n$维傅里叶变换}
$n$维傅里叶变换定义为
\begin{eqnarray}
&& F(\lambda_1, \cdots, \lambda_n) = \mathscr{F}[ f(x_1,\cdots,x_n) ]
\nonumber\\
&=& \int^\infty_{-\infty} \cdots \int^\infty_{-\infty} f(x_1, \cdots, x_n) 
e^{-i(\lambda_1 x_1 + \cdots + \lambda_n x_n )} d x_1 \cdots d x_n.
\end{eqnarray}
逆变公式为
\begin{eqnarray}
&&f(x_1, \cdots, x_n) 
\nonumber\\
&=& 
\frac{1}{ (2\pi)^n } \int^\infty_{-\infty} \cdots \int^\infty_{-\infty}
F(\lambda_1, \cdots, \lambda_n) e^{ i(\lambda_1 x_1 + \cdots + \lambda_n x_n )}
d \lambda_1 \cdots d \lambda_n.
\end{eqnarray}
可以导出与1维情况相似的一系列性质。
\end{frame}

\begin{frame}{$\delta$函数的傅里叶变换}
因为 
\begin{equation}
\mathscr{F}[ \delta(x) ] = \int^\infty_{-\infty} \delta(x) e^{-i\lambda x} dx = 1,
\end{equation}
所以{\bf 约定}:$\mathscr{F}^{-1} [1] = \delta(x)$。
\end{frame}

\begin{frame}{求解弦振动方程初值问题}

\begin{equation}
\left\{
\begin{aligned}
u_{tt} &=& a^2 u_{xx},  ~~ -\infty < x < \infty, t>0, \\
u(x,0) &=& \varphi(x), ~~ -\infty < x < \infty, \\
u_t(x,0) &=& 0, ~~ -\infty < x < \infty.
\end{aligned}
\right.
\end{equation}
取$u(x,t),\varphi(x)$的傅里叶变换,
\begin{equation}
\mathscr{F}[ u ] = \tilde{u}(\lambda,t), ~~ \mathscr{F}[ \varphi ] = \tilde{\varphi}(\lambda).
\end{equation}
那么,在傅里叶变换下,$u_{tt} \rightarrow \tilde{u}_{tt} (\lambda,t), u_{xx} \rightarrow - \lambda^2 \tilde{u}$(原像的导数定理)。
所以在傅里叶变换下,方程变为
\begin{equation}
\left\{
\begin{aligned}
\tilde{u}_{tt} &=& - a^2 \lambda^2 \tilde{u}, \\
\tilde{u}(\lambda,0) &=& \tilde{\varphi}(\lambda), \\
\tilde{u}_t(\lambda,0) &=& 0.
\end{aligned}
\right.
\end{equation}
这个方程很容易求解:$\tilde{u} = \tilde{\varphi}(\lambda) \cos(a\lambda t)$。
\end{frame}

\begin{frame}{弦振动方程初值问题}
\begin{eqnarray}
u &=& \mathscr{F}^{-1} [ \tilde{u} (\lambda, t) ] = \mathscr{F}^{-1}[ \tilde{\varphi}(\lambda) \cos a \lambda t ] \nonumber\\
%&=& \frac{1}{2\pi} \int^\infty_{-\infty} \tilde{\varphi}(\lambda) \cos a \lambda t e^{i\lambda x} d\lambda \nonumber\\
%&=& \frac{1}{4\pi}\int^\infty_{-\infty} [ e^{i \lambda(x+at)} + e^{i\lambda(x-at)} ] \tilde{\varphi}(\lambda) d\lambda \nonumber\\
%&=& \frac{1}{2}[ \varphi(x+at) + \varphi(x-at) ].
\end{eqnarray}
最终结果正是达朗贝尔解。
\end{frame}

\begin{frame}{热传导方程初值问题}
\begin{equation}
\left\{
\begin{aligned}
u_t &= a^2 u_{xx},  &-\infty < x < \infty, t>0, \\
u(x,0) &= \varphi(x), &-\infty < x < \infty.
\end{aligned}
\right.
\end{equation}
在傅里叶变换下,方程变为
\begin{equation}
\left\{
\begin{aligned}
\tilde{u}_t &=& -a^2 \lambda^2 \tilde{u}, \\
\tilde{u}(\lambda, 0) &=& \tilde{\varphi}(\lambda)
\end{aligned}
\right.
\end{equation}
其解非常容易得到:$\tilde{u} = \tilde{\varphi}(\lambda) e^{-a^2 \lambda^2 t}$。
\end{frame}

\begin{frame}{热传导方程初值问题}
所以$u$即$\tilde{u}$的逆变换
\begin{equation}
u(x,t) = \mathscr{F}^{-1}[ \tilde{u}(\lambda,t) ]
= \mathscr{F}^{-1}[ \tilde{\varphi} e^{-a^2 \lambda^2 t} ]
= \varphi(\lambda) \star \mathscr{F}^{-1}[ e^{-a^2 \lambda^2 t} ]
\end{equation}
上式使用了卷积定理。

根据泊松积分(教材(5.24)式),有
\begin{eqnarray}
&& \mathscr{F}^{-1}[ e^{-a^2 \lambda^2 t} ] = \frac{1}{2\pi} \int^\infty_{-\infty} e^{-a^2 \lambda^2 t} e^{i\lambda x} d\lambda
\nonumber\\
&=& \frac{1}{2\pi} \int^\infty_{-\infty} e^{-a^2 \lambda^2 t} ( \cos \lambda x + i \sin \lambda x) d \lambda
\nonumber\\
&=& \frac{1}{\pi} \int^\infty_0 e^{-a^2 \lambda^2 t} \cos \lambda x d \lambda
= \frac{1}{2a \sqrt{\pi t}} e^{-x^2/(4a^2t)}.
\end{eqnarray}
所以最终解为
\begin{equation}
u(x,t) = \varphi(x) \star \frac{1}{2a \sqrt{\pi t}} e^{-x^2/(4a^2t)}
= \frac{1}{2a \sqrt{\pi t}} \int^\infty_{-\infty} \varphi(\xi) e^{-(\xi-x)^2/(4a^2t)} d\xi.
\end{equation}
\end{frame}

\begin{frame}{作业}

\kong[0.5]
课堂选讲:2,4

\kong[0.5]
课后习题:1,3,5

\kong[1]
\end{frame}

\end{document}