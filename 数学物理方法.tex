
\documentclass[cn,hazy,blue,14pt,normal]{elegantnote}
\title{数学物理方法笔记}

\author{Jaden Feng\\冯杰骏}

\date{}

\usepackage{array}
\usepackage{physics}
\usepackage{esint}
\usepackage{amsmath}
\numberwithin{equation}{section}
\usepackage{amssymb}
\usepackage{wrapfig}
\usepackage{tikz}
\usetikzlibrary{arrows, positioning, calc, angles, quotes}

\begin{document}

\maketitle
\pagenumbering{roman}
\setcounter{page}{1}
\newpage
\begin{center}
    \Huge\textbf{{Introduction}}\\
\end{center}~\
数学物理方法笔记。
\begin{flushright}
    \begin{tabular}{c}
        Jaden Feng\\
        冯杰骏\\
        \date{\today}
    \end{tabular}
\end{flushright}

\newpage
\pagenumbering{Roman}
\setcounter{page}{1}
\tableofcontents

\newpage
\setcounter{page}{1}
\pagenumbering{arabic}

\section{复数与复变函数}
\newpage
\subsection{复数}
\begin{definition}
  虚数单位$i$
\end{definition}
$$i^2=-1 $$

\begin{definition}
  复数:
\end{definition}
$$z=x+iy$$ 
\begin{center}
	实部:$x=\text{Re}\,z$ \qquad 虚部:$y=\text{Im}\,z$ (没有 $i$ )
\end{center}

\begin{definition}
	  共轭复数:
\end{definition}
$$\overline{z} =x-iy$$
\begin{definition}
	复平面:
\end{definition}
$x$ 轴为实数轴,$y$ 轴为虚数轴,
在复平面内,用矢量 $\vec{O_z}$ 代表复数 $z=x+iy$​ ,
矢量 $\vec{O_z}$ 的长度为模(绝对值)$r=|z|=\sqrt{x^2+y^2} \geqslant 0 $。\\
\\
几个公式:
\begin{equation}\label{Complex_fundamental_inequality}
	\begin{aligned}
		|x| &\leqslant |z|, \\
		|y| &\leqslant |z|, \\
		|z| &\leqslant |x|+|y|, \\
		|z_1|+|z_2| &\geqslant |z_1 \pm z_2|, \\
		||z_1|-|z_2|| &\leqslant |z_1 \pm z_2| 
	\end{aligned}
\end{equation}

\begin{definition}
	幅角:
\end{definition}
$$\text{Arg}\ z=\theta+2k\pi , k\in\mathbb{Z}$$

幅角主值(主幅角):  $\text{arg}\ z \in (-\pi,\pi]\ or \ [0,2\pi)$\\​

若 $\text{arg}\ z \in (-\pi,\pi]$ ,有:
\begin{equation*}
    \text{arg}\ z=
\left \{
    \begin{array}{lr}
        \arctan\frac{y}{x},& x>0,\\ 
        \frac{\pi}{2},& x=0,y>0,\\
        \arctan\frac{y}{x}+\pi, & x<0,y\geqslant0,\\
        \arctan\frac{y}{x}-\pi, & x<0,y<0,\\
        -\frac{\pi}{2},& x=0,y<0.\\
    \end{array} 
\right.
\end{equation*}
\begin{note}
就是把矢量方向转到一,四象限。
\end{note}
\begin{definition}
	复数的三角形式:
\end{definition}
利用极坐标,复数还可以写成三角形式。
$$
z=x+iy=r(\cos\theta+i\,\sin\theta)
$$
$r=1$ 时,$z=\cos\theta+i\,\sin\theta$​ 称为单位复数
\begin{definition}
	复数的指数形式:
\end{definition}
由欧拉公式 $e^{i\theta}=\cos\theta+i\,\sin\theta$​​ :
$$
z=x+iy=r(\cos\theta+i\,\sin\theta)=re^{i\theta}
$$
$z=re^{i\theta}$ 称为指数形式
\begin{note}
	如何将普通形式改写为三角形式和指数形式?\\
	只需提取公因式 $r=\sqrt{x^2+y^2}$​ 即可, 需要注意的是:$r>0$ 。
\end{note}
指数形式下容易验证:
$$
\left \{
	\begin{array}{l}
		z_1\,z_2=r_1r_2e^{i(\theta_1+\theta_2)},\\
		\frac{z_1}{z_2}=\frac{r_1}{r_2}e^{i(\theta_1-\theta_2)}.
	\end{array}
\right .
$$
这表明:
$z_1\,z_2$ 对应的矢量就是把 $z_1$ 伸缩 $|z_2|$ 倍,然后幅角加 $\theta_2$ (逆时针旋转 $\theta_2$ ) ; $\frac{z_1}{z_2}$ 同理。
\begin{definition}
	棣莫弗公式:
\end{definition}
考虑复数 $z$ 的正整数次幂 $z^n$ :
\begin{equation}
	z^n=r(\cos\theta+i\,\sin\theta)^n=r^ne^{in\theta}=r^n(\cos n\theta+\sin n\theta)
\end{equation}

$r=1$ 时,得到棣莫弗公式:
\begin{equation}\label{De Moivre's formula}
(\cos\theta+i\,\sin\theta)^n=\cos n\theta+\sin n\theta
\end{equation}
\begin{example}
	计算三倍角公式
\end{example}
由棣莫弗公式\ref{De Moivre's formula}:
 $$
(\cos\theta+i\,\sin\theta)^3=\cos 3\theta+i\sin3\theta
$$
\indent 用二项式定理 $(a + b)^n = \sum\limits_{k=0}^{n} C_{n}^{k} a^{n-k} b^k$ 展开左边:

\begin{align}
	&\cos^3\theta+3i\cos^2\theta\sin\theta-3\cos\theta\sin^2\theta-i\sin^3\theta
 	\\
	=&(\cos^3\theta-3\cos\theta\sin^2\theta)+i(3\cos^2\theta\sin\theta-\sin^3\theta)
	\\
	=&\cos 3\theta+i\sin3\theta
\end{align}

实部对应实部,虚部对应虚部:

\begin{align}
 	&\cos 3\theta=\cos^3\theta-3\cos\theta\sin^2\theta=4\cos^3\theta-3\cos\theta,
	\\
	&\sin3\theta=3\cos^2\theta\sin\theta-\sin^3\theta=3\sin\theta-4\sin^3\theta.
\end{align}

\begin{note}
	两式中第二个等号都用了 $\cos^2\theta+\sin^2\theta=1$
\end{note}
\begin{definition}
	复数的n次方根:
\end{definition}
下面求复数的n次方根:
假设 $w^n=z,\;i.e.\ w=\sqrt{z}$ ,其中 $w=\rho e^{i\varphi},z=re^{i\theta}$​ ,有:
\begin{gather}
	\rho^n=r,\quad n\varphi=\theta+2k\pi;\\
\rho=\sqrt{r},\quad \varphi=\frac{\theta+2k\pi}{n}.
\end{gather}
所以:
$$
w_k=(\sqrt[n]{z})_k=\sqrt[n]{r}e^{i\frac{\theta+2k\pi}{n}}=\sqrt[n]{r}e^{i\frac{\theta}{n}}e^{i\frac{2k\pi}{n}}
$$
\begin{note}
	$z_1\,z_2$ 对应的矢量就是把 $z_1$ 伸缩 $|z_2|$ 倍,然后幅角加 $\theta_2$ (逆时针旋转 $\theta_2$ ) ; $\frac{z_1}{z_2}$ 同理。
与这句话原理相同,$e^{i\frac{2k\pi}{n}}$ 的作用效果是把 $\sqrt[n]{r}e^{i\frac{\theta}{n}}$ 的幅角加 $\frac{2k\pi}{n}$ (逆时针旋转 $\frac{2k\pi}{n}$ ) 。
$k$ 从 $0$ 开始取,取到 $n$ 时转回原点,所以共有 $n-1$ 个根。定义 $w_0=\sqrt[n]{r}e^{i\frac{\theta}{n}}$ ,可以得到 $w_k=w_0e^{i\frac{2k\pi}{n}},\quad k\in[0,n-1]$​。\\
实际计算中可以直接计算 $\frac{1}{n}$ 次方,但是要把被开方数的幅角写成 $\theta+2k\pi$ 的形式。
\end{note}


\newpage
\section{解析函数}
\newpage
\subsection{解析函数的概念以及C-R条件}

\begin{definition}
	C-R条件:
\end{definition}
直角坐标系下,对于复变函数$f(z)=u(x,y)+iv(x,y)$。
\begin{gather}
	\frac{\partial u}{\partial x}=\frac{\partial v}{\partial y},\qquad
	\frac{\partial u}{\partial y}=-\frac{\partial v}{\partial x}.
\end{gather}
称为柯西-黎曼条件(方程)。
\par 极坐标系下,对于复变函数$f(z)=u(r,\theta)+iv(r,\theta)$。
\begin{gather}
	\frac{\partial u}{\partial r}=\frac{1}{r}\frac{\partial v}{\partial \theta},\qquad
	\frac{1}{r}\frac{\partial u}{\partial \theta}=-\frac{\partial v}{\partial r}.
\end{gather}
\begin{note}
	记忆方法:$\theta$没有量纲,所以需要乘$\frac{1}{r}$让两边量纲一致。
\end{note}
\begin{theorem}
	函数$f(z)=u(x,y)+iv(x,y)$在区域$D$的内点$z=x+iy$可微的充分必要条件是$u(x,y),v(x,y)$在$(x,y)$处可微且满足C-R条件。
\end{theorem}
\begin{proof}

\end{proof}
\textbf{必要性}:i.e. $f(z)$可微$\to u,v $可微,满足C-R条件\\
如果$f(z)$在$D$内$z$点可微,那么
$$
	\Delta f(z)=f(z+\Delta z)-f(z)=f'(z)\Delta z+\rho(\Delta z)
$$
$\rho(\Delta z)$是$\Delta z$的高阶无穷小。i.e. $\lim\limits_{\Delta z\to0}\frac{\rho(\Delta z)}{\Delta z}=0$。\\ 
令$f'(z)=a+ib$,则:
\begin{align*}
	f(z+\Delta z)-f(z)&=[(u+\Delta u)+i(v+\Delta v)]-(u+iv)\\
	&=\Delta u + i \Delta v\\
	f'(z)\Delta z+\rho(\Delta z)&=a+ib(\Delta x+i\Delta y)+\rho(\Delta z)\\
	&=a\Delta x-b\Delta y+i(b\Delta x+a\Delta y)+\eta_1 + i\eta_2
\end{align*}
所以:
$$
\Delta u + i \Delta v=a\Delta x-b\Delta y+i(b\Delta x+a\Delta y)+\eta_1 + i\eta_2
$$
比较实部和虚部:
\begin{gather*}
	\Delta u=a\Delta x-b\Delta y+\eta_1,\\
	\Delta v=b\Delta x+a\Delta y+\eta_2.
\end{gather*}
由二元实函数微分的定义:
\emph{如果$f(x,y)$的自变量$x,y$在点$(x_0,y_0)$分别取得改变量$\Delta x,\Delta y$,
如果全改变量$\Delta f=f(x_0+\Delta x,y_0+\Delta y)-f(x_0,y_0)$可以表示为两个部分之和:
一部分是一个线性式,也就是$f(x,y)$的全微分$df=A\Delta x+ B\Delta y$,
另一部分是$\sqrt{{\Delta x}^2+{\Delta y}^2}$的高阶无穷小。那么就说$f(x,y)$在点$(x_0,y_0)$可微。}\\
所以$\Delta u,\Delta v$在$(x,y)$处可微。\\
分别对这两个式子求偏导数可以得到$u_x=a=v_y$,$u_y=-b=-v_x$。这就是直角坐标系下的C-R条件。\\

\par \textbf{充分性}:i.e. $u,v$可微,满足C-R条件$\to f(z)$可微\\
要证明$f(z)$可微,就要证明$\Delta f(z)$可以表示为一个线性式和一个高阶无穷小的和。
i.e. $\Delta f(z)=A(z)\Delta z+\rho(\Delta z)$\\
显然:$$\Delta f(z)=\Delta u+i\Delta v$$
又因为$u,v$可微,所以:
(这里是由二元实函数微分的定义得到的,此定义已经在必要性的证明过程中写出。)
\begin{gather*}
	\Delta u=u_x\Delta x+u_y\Delta y+\eta_1(\Delta x,\Delta y),\\
	\Delta v=v_x\Delta x+v_y\Delta y+\eta_2(\Delta x,\Delta y).
\end{gather*}
又因为$u,v$满足C-R条件,所以:
\begin{gather*}
	u_x=v_y=\alpha,\\
	u_y=-v_x=-\beta
\end{gather*}
代入$\Delta f(z)$:
\begin{align*}
	\Delta f(z)&=\Delta u+i\Delta v\\
	&=(u_x\Delta x+u_y\Delta y+\eta_1)+i(v_x\Delta x+v_y\Delta y+\eta_2)\\
	&=(\alpha+i\beta)(\Delta x+i\Delta y)+\eta_1+i\eta_2\\
	&=(\alpha+i\beta)\Delta z+\rho(\Delta z)
\end{align*}
回过头去看看复变函数微分的定义:一个线性式和一个高阶无穷小的和。i.e. $\Delta f(z)=A(z)\Delta z+\rho(\Delta z)$\\
显然只要证明$\rho$是无穷小量就可以了。(关于$z$的)\\
由公式(\ref{Complex_fundamental_inequality})的第二个式子可以得到:
$$
\left\lvert \frac{\rho}{\Delta z} \right\rvert=\left\lvert \frac{\eta_1+i\eta_2}{\Delta z} \right\rvert
\leqslant\left\lvert \frac{\eta_1}{\Delta z} \right\rvert+\left\lvert \frac{\eta_2}{\Delta z} \right\rvert
$$
当$\Delta z \to 0$时,因为$u,v$都可微,所以不等号右边两项都等于0(还是因为定义),
所以$\left\lvert \frac{\rho}{\Delta z} \right\rvert=0$,i.e. $\rho$是无穷小量。\\
综上,我们证明了$f(z)$可微。\\
同时我们还证明了:$f'(z)=\alpha+i\beta=u_x+iv_x$。\\
证毕。

\begin{example}
	举一个满足C-R条件,但是不可微的例子:
	$f(z)=\sqrt{|xy|}$在$z=0$处满足C-R条件,但是不可微。
\end{example}
函数$f(z)=\sqrt{|xy|}=u(x,y)+iv(x,y)$。于是:\\
\begin{gather*}
	u_x(0,0)=\lim_{\Delta x\to0}\frac{u(\Delta x,0)-u(0,0)}{\Delta x}=\lim_{\Delta x\to0}\frac{\sqrt{|\Delta x\cdot0|}}{\Delta x}=0=v_y(0,0),\\
	u_y(0,0)=\lim_{\Delta y\to0}\frac{u(0,\Delta y)-u(0,0)}{\Delta y}=\lim_{\Delta y\to0}\frac{\sqrt{|\Delta y\cdot0|}}{\Delta y}=0=-v_x(0,0).\\
\end{gather*}
满足C-R条件。但是:\\
\begin{equation*}
	\frac{f(\Delta z)-f(0)}{\Delta z}=\frac{\sqrt{\Delta x \cdot \Delta y}}{\Delta x+i\Delta y}
\end{equation*}
当沿$\Delta y=k\Delta x$趋近于0时,有:原式$=\frac{\sqrt{k}}{1+ik}$与$k$有关,所以原式不可微。

\begin{definition}
	解析点
\end{definition}
如果函数$\omega=f(z)$在点$z_0$的某邻域内处处可微,那么就说$z_0$是函数$\omega=f(z)$的解析点,
或者说函数$\omega=f(z)$在点$z_0$解析。

\begin{definition}
	解析函数
\end{definition}
如果区域$D$内的每一点都是函数$\omega=f(z)$的解析点,那么就说函数$\omega=f(z)$在区域$D$内解析,
或者说函数$\omega=f(z)$是区域$D$内的解析函数。

\begin{definition}
	奇点
\end{definition}
如果$f(z)$在$z_0$点不解析,但在$z_0$的邻域总有一点解析,那么就说$z_0$是$f(z)$的奇点。

\begin{example}
	$f(z)=e^x(\cos y+i\sin y)$在$z$平面上解析,而且$f'(z)=f(z)$
\end{example}

把原式展开可以知道:
$$
u(x,y)=e^x\cos y,\quad v(x,y)=e^x\sin y
$$
所以:
\begin{gather*}
	u_x=e^x\cos y,\quad u_y=-e^x\sin y\\
	v_x=e^x\sin y,\quad v_y=e^x\cos y
\end{gather*}
这几个式子证明了$u,v$的偏导数存在,又因为$u,v$都连续(初等函数的组合),所以$u,v$可微。\\
这几个式子又满足C-R条件。所以$f(z)$在$z$平面上解析。\\

\begin{example}
	由条件:
	$\left\{
	\begin{aligned}
		&u(x,y) = x^2-y^2+xy\\
		&f(i) = -1+i
		\end{aligned}
	\right.$
	求解析函数$f(z)=u+iv$
\end{example}
因为是解析函数,所以满足C-R条件:$u_x=v_y$,$u_y=-v_x$。\\
也就是:
\begin{gather*}
	2x+y=v_y,\\
	-x+2y=v_x.
\end{gather*}
从上面的第一个式子可以知道:
$$
v=\int (2x+y)dy + v(x) = 2xy+\frac{1}{2}y^2 + \varphi(x) + C
$$
\textbf{这里的$\varphi(x)$是$v$里面只有$x$的项}\\
把这个式子对$x$求导,可以得到:(这里是因为对$x$求导可以用到剩下的那个式子,所以想到的。)
$$
v_x=2y+\varphi'(x)=2y-x
$$
所以
$$
\varphi'(x)=-x\qquad\Rightarrow\qquad\varphi(x)=-\frac{1}{2}x^2
$$
所以:
$$
f(x,y)=x^2-y^2+xy+i(2xy+\frac{1}{2}y^2-\frac{1}{2}x^2+C)
$$
我们又知道,$f(i)=-1+i$,所以把$i$带进去($x=0,y=1$):
$$
-1+i=-1+i(\frac12 + C) \qquad\Rightarrow\qquad C=\frac12
$$
所以:
$$
f(x,y)=x^2-y^2+xy+i(2xy+\frac{1}{2}y^2-\frac{1}{2}x^2+\frac12)
$$
我们希望把这个式子化成$f(z)$而不是$f(x,y)$的形式。
这需要利用共轭复数。\\
我们知道
$$
z=x+iy,\quad \overline{z}=x-iy
$$
经过一些很容易的变换(两式相加,两式相减),可以得到:
$$
x=\frac{z+\overline{z}}{2},\quad y=\frac{z-\overline{z}}{2i}
$$
把这个结果带进去,可以得到:
$$
f(z)=(1+\frac i2)z^2+\frac i2
$$
\begin{note}
	这里需要注意两点:
	\begin{itemize}
		\item $\varphi(x)$是$v$里面只有$x$的项
		\item 利用共轭复数,将$f(x,y)$化成$f(z)$的形式
	\end{itemize}
\end{note}
\newpage
\subsection{解析函数与调和函数的关系}


\end{document}
